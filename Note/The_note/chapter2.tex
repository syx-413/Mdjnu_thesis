\section{有限维向量空间}
\subsection{张成与线性无关}
\begin{definition}
    V中一组向量$(v_{1}...,v_{m})$的线性组合是如下的向量
    \begin{equation}
     a_{1}\textbf{v}_{1}+...+a_{m}\textbf{v}_{m},a_{1},...,a_{m}\in F   
    \end{equation}
\end{definition}
其中$向量(v_{1},...,v_{m})$所有的线性组合构成的集合称为$(v_{1},...,v_{m})$的张成(span)
记为$span(v_{1},...,v_{m})$即为
\begin{equation}
    span(\textbf{v}_{1},...,\textbf{v}_{m})=\left\{ a_{1}\textbf{v}_{1}+...+a_{m}\textbf{v}_{m}:a_{1},...,a_{m}\in  F\right\}
\end{equation}
\par 其中其中空组()的张成空间为\{0\} 
\begin{definition}
    如果$span(v_{1},...,v_{m})$ 等于 V,则称$s(v_{1},...,v_{m})$张成 V,
\end{definition}
其中:转到多项式的定义,对于多项式 $p\in\mathcal{P}(F)$ 如果存在标量 $a_{0},a_{1},...,a_{m}\in F,a_{m}\neq 0,使得$
\begin{center}    
$p(z)=a_{0}+a_{1}z+...+a_{m}z^m,z\in F$
\end{center}
\par 则说 $p$ 的次数(degree),为 m.规定恒等于0的次数为 $-\infty$
\subsection*{线性无关}
\begin{definition}
    对于 V 中的一组$(v_{1},...,v_{m})$,如果使得 $(a_{1}\textbf{v}_{1}+...+a_{m}v_{m})=0$ 
的$a_{1}...a_{m}\in F$,只有$a_{1}=...=a_{m}= 0$,则称$(v_{1},...,v_{m})$ 
是{\bfseries 线性无关}的。
\end{definition}
\subsection{基}
\begin{definition}
若 V 中的一个向量既是线性无关的,又张成 V,则称之为V 的{\bfseries 基}(basis)
\end{definition}
这是个$\textbf{F}^n$的一个基,称为$\textbf{F}^n$的标准正交基:\\
\begin{equation*}    
\left ( \left ( 1,0,...,0 \right ) ,\left (0,1,...,0  \right ),...\left ( 0,0,...,1 \right )\right )
\end{equation*}
\begin{definition}
    V中向量组$(v_{1},...,v_{n} )$ ,是V的基当且仅当每一个$\textbf{v}\in V$都能写成如下\\
    \begin{center}
        $\textbf{v}=(a_{1}\textbf{v}_{1},...,a_{n}\textbf{v}_{n})$,其中$a_{1},…,a_{n}\in F$   
    \end{center}
\end{definition}

\begin{definition}
    在向量空间中,每个张成组都可以简化为一个基
\end{definition}
\begin{definition}
    推论:每个有限维向量空间都有基
\end{definition}
\begin{definition}
    在有限维向量空间中,每一个线性无关向量组都可以扩充成一个基
\end{definition}
\begin{definition}
    设V是有限维度,U是V的一个子空间,则存在V的一个子空间$W$使得$V=U\oplus W$
\end{definition}
\subsection{维数}
\begin{definition}
    有限维向量空间的任意两个基长度是相同的(定义$F^n$的维数为n,其中基的长度也为n)
\end{definition}
\begin{definition}
    有限维向量空间的任意基长度称为这个向量空间的{\bfseries 维数}(dimension),V的维数记为dimV

例如:$dim F^n=n$ \qquad $dim\mathcal{P}_{m}(F)=m$+1
\end{definition}
\begin{definition}
    若V是有限维的,并且U是V的子空间,则$ dimU\leq dimV$
\end{definition}
\begin{definition}
    若V是有限维的,则V中的每一个长度dimV的张成向量组都是V的一个基
\end{definition}
\begin{definition}
    如果V是有限维的,则V中每个长度dimV的线性无关向量组都是V的基
\end{definition}
\begin{definition}
    如果$U_1$和$U_2$是同一个有限维向量空间的两个字空间则
    \begin{center}     
$dim(U_1+U_2)=dimU_1+dimU_2-dim\left ( U_1\cap U_2\right )$
    \end{center}
\end{definition}

\begin{definition}
    设V是有限维的,并且$U_1,…,U_m$是V的子空间,使得$V=U_1+…+U_m$,
    \begin{center}     
$dimV=dimU_1+…+dimU_m,则{\color{red}V=U_1\oplus … \oplus U_m}$
    \end{center}
\end{definition}