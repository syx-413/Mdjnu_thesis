% 以下是符合要求的LaTeX代码:




\documentclass{article}
\usepackage[a4paper, left=3cm, right=2cm, top=2.5cm, bottom=2.5cm]{geometry} % 设置纸张大小和页边距
\usepackage{setspace} % 行间距宏包

% 使用16开纸裁切后的尺寸
\setlength{\paperwidth}{184mm}
\setlength{\paperheight}{260mm}

% 使用单面打印
\usepackage{pdfpages}

\begin{document}
% 设置行距
\doublespacing

% 在此插入论文内容
% ...


\end{document}

% --------------------------------------------------------------

% ```latex
\documentclass{article}
\usepackage{ctex} % 中文支持
\setmainfont{Times New Roman} % 英文字体
% 设置中文字体
\setCJKmainfont{SimSun} % 宋体
\setCJKsansfont{SimHei} % 黑体
\setCJKmonofont{FangSong} % 仿宋
% 定义各级标题格式
\usepackage{titlesec}
\titleformat{\section}[hang]{\heiti\zihao{4}}{\thesection}{1em}{}
\titleformat{\subsection}[hang]{\heiti\zihao{-4}}{\thesubsection}{1em}{}
\titleformat{\subsubsection}[hang]{\heiti\zihao{-4}}{\thesubsubsection}{1em}{\bfseries}
% 设置正文段落格式
\setlength{\parindent}{2em} % 首行缩进
\linespread{1.5} % 行距
% 定义目录格式
\usepackage[titles]{tocloft}
\renewcommand{\cfttoctitlefont}{\heiti\zihao{3}\bfseries}
\renewcommand{\cftsecfont}{\heiti\zihao{4}}
\renewcommand{\cftsubsecfont}{\heiti\zihao{-4}}
\renewcommand{\cftsubsubsecfont}{\heiti\zihao{-4}\bfseries}
\renewcommand{\cftsecpagefont}{\heiti\zihao{4}}
\renewcommand{\cftsubsecpagefont}{\heiti\zihao{-4}}
\renewcommand{\cftsubsubsecpagefont}{\heiti\zihao{-4}\bfseries}

% 定义附录、致谢、独创性声明格式
\titleformat{\chapter}[hang]{\heiti\zihao{3}\bfseries}{\chaptername}{1em}{}
\titlespacing{\chapter}{0pt}{-3em}{2.5em}
\newenvironment{acknowledge}{\heiti\zihao{3}\bfseries 致谢}{}
\newenvironment{appendix}{\chapter*{\heiti\zihao{3}\bfseries 附录}}{}
\newenvironment{declaration}{\chapter*{\heiti\zihao{3}\bfseries 独创性声明}}{}

\begin{document}

% 论文题目
\title{\heiti\zihao{2} 这是论文题目}
\author{\kaishu\zihao{-3}作者姓名}
\date{\zihao{-4}\today}
\maketitle

% 目录
\tableofcontents

% 摘要
\section*{\heiti\zihao{3}\bfseries 摘要}
这是摘要的内容。

% 正文
\section{\heiti\zihao{4} 第一章}
这是第一章的内容。

\subsection{\heiti\zihao{-4} 第一节}
这是第一节的内容。

\subsubsection{\heiti\zihao{-4}\bfseries 第一小节}
这是第一小节的内容。

% 结论
\section*{\heiti\zihao{3}\bfseries 结论}
这是结论的内容。

% 参考文献
\section*{\heiti\zihao{3}\bfseries 参考文献}
\begin{thebibliography}{99}
    \bibitem{ref1} 作者. 文章题目[J]. 期刊名, 年份, 卷号(期号): 起止页码.
\end{thebibliography}

% 附录
\begin{appendix}
    \chapter*{\heiti\zihao{3}\bfseries 附录}
    这是附录的内容。
\end{appendix}

% 致谢
\begin{acknowledge}
    这里是致谢的内容。
\end{acknowledge}

% 独创性声明
\begin{declaration}
本人声明所呈交的学位论文是本人在导师指导下进行的研究